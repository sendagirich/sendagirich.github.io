% Created 2023-07-22 Sat 22:08
% Intended LaTeX compiler: pdflatex
\documentclass[11pt]{article}
\usepackage[utf8]{inputenc}
\usepackage[T1]{fontenc}
\usepackage{graphicx}
\usepackage{longtable}
\usepackage{wrapfig}
\usepackage{rotating}
\usepackage[normalem]{ulem}
\usepackage{amsmath}
\usepackage{amssymb}
\usepackage{capt-of}
\usepackage{hyperref}
\author{Richard Williams}
\date{\today}
\title{My School Day}
\hypersetup{
 pdfauthor={Richard Williams},
 pdftitle={My School Day},
 pdfkeywords={},
 pdfsubject={},
 pdfcreator={Emacs 28.2 (Org mode 9.6–9.6-??-bed47b4)}, 
 pdflang={English}}
\usepackage{biblatex}
\addbibresource{~/Dropbox/org/bibliography/references.bib}
\begin{document}

\maketitle
\tableofcontents

\begin{center}
\begin{tabular}{}
\\\empty
\end{tabular}
\end{center}
\section*{Web page}
\label{sec:orgdeb4cea}
\subsection*{Lessons}
\label{sec:org18b2638}
\subsubsection*{Week one. September 4th}
\label{sec:orge41d6df}
\texttt{You will learn:}
\begin{verbatim}
How to give you opinion on school subjects.

The course plan and project.

How to use the course web-page in English.
\end{verbatim}
\texttt{Example English}

\begin{itemize}
\item \textbf{A.} ``\emph{What do you think about math?}''
\item \textbf{B.} ``\emph{Math is difficult but interesting.}''
\item \textbf{A.} ``\emph{I see.}''
\end{itemize}
\texttt{Homework}
\begin{verbatim}
 Review the worksheet.
 Look at website:
   Find...
\end{verbatim}

\texttt{Materials}
\begin{itemize}
\item \href{https://sendagirich.github.io/school\_wks.pdf}{worksheets}
\item website language (link)
\end{itemize}
\subsection*{Course plan}
\label{sec:org39092c8}

\texttt{Course aims}
\begin{verbatim}
You will learn English to talk about school life.

You will learn English to learn, and talk about film making.

In groups, you will make a short video about your school (in English).
\end{verbatim}
\begin{center}
\begin{tabular}{}
\\\empty
\end{tabular}
\end{center}

\texttt{Course plan}

\begin{center}
\begin{tabular}{rlll}
\hline
 &  & English & Project\\\empty
\hline
1. & 09/07 & Intro project. School subjects & Intro film project and web-page\\\empty
2. & 09/14 & School periods & Film periods - future, past\\\empty
3. & 09/21 & School places \& activities & Setting, location\\\empty
4. & 09/28 & Prepositions of place & Film shots \& composition\\\empty
5. & 10/?? & School rules & Film making do's and don'ts\\\empty
6. & 10/?? & Transport, coming to school & Intro video making watch vid\\\empty
7. & 10/?? & Structuring a story & Write script. take photos for homework\\\empty
8. & 10/?? & Review & Submit video\\\empty
9. & 11/?? & \textbf{TEST} & Video review. Criticism language\\\empty
10. & 11/?? & Suggestions & Constructive criticism / film trailer\\\empty
12. & 11/?? & Film genre specific language & Film trailer\\\empty
13. & 11/?? & Film genre specific language & Film trailer\\\empty
14. & 12/?? &  & Stand alone lesson.\\\empty
\hline
\end{tabular}
\end{center}



\texttt{Assessment}


You will have a speaking test where you will be asked questions about your school life. Information about assessment is here: \href{https://sendagirich.github.io/assessment.html}{Assessment}

\noindent\rule{\textwidth}{0.5pt}

\begin{HTML}
\href{https://sendagirich.github.io/school-project/school\_materials.html}{School Materials}
\begin{center}
\begin{tabular}{}
\\\empty
\end{tabular}
\end{center}
\href{https://sendagirich.github.io/school-project/extra\_materials.html}{Extra Material}
\begin{center}
\begin{tabular}{}
\\\empty
\end{tabular}
\end{center}
\href{https://sendagirich.github.io/school-project/assessment.html}{Assessment}
\begin{center}
\begin{tabular}{}
\\\empty
\end{tabular}
\end{center}
\href{https://sendagirich.github.io/school-project/shots.html}{Shots}
\end{HTML}
\end{document}